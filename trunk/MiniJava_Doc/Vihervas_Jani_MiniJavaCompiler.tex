\documentclass[a4paper,12pt]{article}
\usepackage[utf8]{inputenc}
\usepackage[T1]{fontenc}
\usepackage[english]{babel}
\usepackage[charter]{mathdesign}
\usepackage{beton}
\renewcommand{\bfdefault}{bx}
\renewcommand{\scdefault}{sc}
\usepackage{textcomp}
\usepackage{hyperref}
\usepackage{enumerate}
\usepackage{amsmath}
\usepackage[all]{xy}
\usepackage[left=1.5cm,right=1.5cm,top=1.5cm,bottom=1.5cm]{geometry}
\setlength{\parindent}{0pt}
\setlength{\parskip}{1em}

\author{Jani Viherväs}
\hypersetup{pdfinfo={
Title={Mini-Java Compiler},
Author={Jani Viherväs},
Subject={Code Generation},
Keywords={compiler, Mini-Java, Code Generation}
}}


\newcommand{\ii}[1]{\textit{#1}}
\newcommand{\bb}[1]{\textbf{#1}}
\newcommand{\ttt}[1]{\texttt{#1}}
\newcommand{\e}{\epsilon}
\newcommand{\s}{\text{ }}
\newcommand{\integer}{\ttt{i}}
\newcommand{\str}{\ttt{s}}
\newcommand{\bool}{\ttt{b}}
\newcommand{\bracel}{\textbraceleft}
\newcommand{\bracer}{\textbraceright}


\begin{document}
\thispagestyle{empty} 
\begin{flushright}
\today \\
\vspace{1em}
Jani Viherväs\\ 
\href{mailto:jani.vihervas@cs.helsinki.fi}{jani.vihervas@cs.helsinki.fi}
\end{flushright}

\vfill

\begin{center}
\textsc{\LARGE Mini-Java Compiler} \\
\vspace{1em}
\textsc{\large 582648 Code Generation}\\
\vspace{1em}
\url{http://www.cs.helsinki.fi/u/vihavain/k12/compiler_project/project/compiler_project_2012.html}
\end{center}

\vfill
% \pagebreak

\setcounter{page}{1}

\section{Overview}

\section{Grammar}
\begingroup\footnotesize
\begin{align*}
<program>\s \to \s &<main\s class> <class\s declaration>* \\
<main\s class>\s \to \s &\bb{class} <new\s identifier> \bb{\bracel public static void main() \bracel} <statement>* \bb{\bracer \bracer}\\
<class\s declaration>\s \to \s &\bb{class} <new\s identifier> [\bb{extends} <identifier>] \bb{\bracel}<declaration>* \bb{\bracer}\\
<declaration>\s \to \s &<variable\s declaration> | <method\s declaration> \\ 
% \vspace{-2mm}\\
\hline
<method\s declaration>\s \to \s &\bb{public} <type> <new\s identifier> \bb{(} [ <formals> ] \bb{) \bracel} <statement>* \bb{\bracer}\\
<variable\s declaration>\s \to \s &<type> <new\s identifier> <variable\s assignment>\bb{;}\\
<variable\s assignment>\s \to \s &\epsilon \mid = <expr>\\
<formals>\s \to \s &<type> <new\s identifier> ( \bb{,} <type> <new\s identifier> )*\\
<type>\s \to \s &<simple\s type> <array\s type>\\
<simple\s type>\s \to \s &\bb{int} \mid \bb{boolean} \mid \bb{void} \mid <type\s identifier>\\
<array\s type>\s \to \s &\epsilon \mid \bb{[ ]}\\
<type\s identifier>\s \to \s &<identifier> \\
% \vspace{-2mm}\\
\hline
<statement>\s \to \s &\bb{assert (} <expr> \bb{);}  \\
             \mid \s &<local\s variable\s declaration>  \\
             \mid \s &\bb{\bracel <statement>* \bracer}  \\
             \mid \s &\bb{if (} <expr> \bb{)} <statement> <else> \\
             \mid \s &\bb{while (} <expr> \bb{)} <statement>  \\
             \mid \s &\bb{System.out.println (} <expr> \bb{);}  \\
             \mid \s &<identifier>[\bb{[}<epxr>\bb{]}] \bb{=} <expr> \bb{;}  \\
             \mid \s &\bb{return} <expr> \bb{;}  \\
             \mid \s &<method\s invocation> \bb{;} \\
<else>\s \to \s &\epsilon \mid \bb{else} <statement> \\
<local\s variable\s declaration> \s \to \s &<variable\s declaration> \\
<method\s invocation>\s \to \s &<expr1> \bb{.} <method\s tail>  \\
<method\s tail>\s \to \s &\bb{length} \mid <identifier> \bb{(} [ <expr> (\bb{,} <expr>)* ] \bb{)} \\
% \vspace{-2mm}\\
\hline
<expr>\s \to \s &<expr1><expr2> \\
<expr1>\s \to \s &\bb{new} <new>\\
    \mid \s    &\bb{!} <expr> \\
    \mid \s    &\bb{-} <expr> \\
    \mid \s    &\bb{(} <expr> \bb{)} \\
    \mid \s    &<identifier> \mid <integer\s literal> \\
    \mid \s    &\bb{this} \mid \bb{true} \mid \bb{false} \\
<expr2>\s \to \s &\epsilon \mid \bb{[} <expr> \bb{]} \mid \bb{.}<method\s tail>\\
    \mid \s    &<binary\s operator> <expr> \\
<new>\s \to \s &<simple\s type> \bb{[} <expr> \bb{]} \mid <type\s identifier> \bb{( )} \\ 
<binary\s operator>\s \to \s & \&\& \mid || \mid < \mid > \mid == \mid + \mid - \mid * \mid / \mid \% 
\end{align*}
\endgroup
\end{document}
